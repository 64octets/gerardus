 

\label{_intro}
\subsection*{Short introduction}

This documentation contains a brief reference manual for the {\bfseries SINTEF} {\bfseries Multilevel} {\bfseries B-\/spline} {\bfseries Library} developed at {\tt SINTEF Applied Mathematics}. The main interface is through the class {\tt MBA} which also contains a small example in its documentation. Note that only the non-\/adaptive/non-\/parametric version of the algorithms are documented here. Parametric and adaptive versions, as used to generate the head-\/model in the figure above, are implemented in the classes MBApar (parametric), MBAadaptive (adaptive) and MBAadaptivePar (adaptive and parametric). 

For a thorough description of the basic schemes, the papers by the originators of Multilevel B-\/splines, S. Lee, G. Wolberg and S. Y. Shin, should be consulted. 

A detailed description of the mathematical content of the algorithms, with extensions made in the SINTEF library and with numerical examples, can be found in the report:\par
 {\itshape �. Hjelle. Approximation of Scattered Data with Multilevel B-\/splines. SINTEF report, 2001.\/}\par
 {\tt Full report (780 K, pdf)} 

\label{_gettingstarted}
\subsection*{Getting started}

It's very easy -\/ just look at the small {\tt example} in class {\tt MBA} and read the documentation for the functions that are called there. Then, copy the complete {\tt main program} to a file -\/ compile it and run it! The program samples the resulting spline surface and writes the result to a VRML-\/file. You can download a free VRML-\/viewer from {\tt Kongsberg SIM}. 

If you have got the Visual C++ workspace with all the source code, then you can just build the application and run it with the current main program. 

\label{_examples}
\subsection*{Numerical examples}

I will make numerical examples and discuss pros and cons thoroughly later. Consult the SINTEF report for mathematical details. 

\label{_furtherWork}
\subsection*{Further work}

Note that the basic algorithms in this library does not solve \char`\"{}the
    scattered data approximation problem\char`\"{} in general. The algorithms will produce anomalies near the data if the underlying grid of B-\/spline coefficients is dense and the data points are unevenly distributed in the domain. The remedy, as far as I have concluded through numerical experiments, is to combine the basic schemes with smoothing operators. This is now being implemented and will be available in the next version. 

\label{_download}
\subsection*{Download}

A GPL-\/version of the library (for Linux/Unix) can be downloaded from {\tt http://www.sintef.no/math\_\-software}. 

Please report any problems or comments to {\tt jan.b.thomassen@sintef.no}. 

�yvind Hjelle, June 2001 

{\itshape Last modified 28.11.2007 by {\tt Jan Thomassen}\/}

{\itshape   \/}